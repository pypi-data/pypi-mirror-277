%  Macros pour usage avec LATEX.  (Pierre L'Ecuyer).

\def\?{\discretionary{}{}{}}  % Same as \- but does not print the - sign
                              % (useful to hyphenate in math mode).
\def\From {From }
\def\mod{{\rm\ mod\ }}        % Modulo.
\def\MOD{{\rm\ MOD\ }}
\def\div{{\rm\ div\ }}        % Division entiere.
\def\DIV{{\rm\ DIV\ }}
\def\var{{\rm var }}        % Variance.
\def\Var{{\rm Var }}
\def\Cov{{\rm Cov }}
\def\MSE{{\rm MSE }}
\def\arg{{\rm\ arg}}          % Argument.
\def\Re{{\rm I\kern-0.2em R}} % Ensemble des nombres reels.
\def\R{\Re}
\def\d{{\rm d}}               % Pour les derivees (parfois).
\def\B{{\rm I\kern-0.2em B}}
\def\F{{\rm I\kern-0.2em F}}
\def\FF{{\rm I\kern-0.1em F}}
\def\N{{\rm I\kern-0.2em N}}  % Ensemble des nombres naturels.
\def\Z{{\sf Z\kern-0.4em Z}}  % Ensemble des nombres entiers.
\def\Fbar{\overline F}        % F avec une barre au dessus.
\def\square{\vrule height6pt width5pt depth1pt}
     %  Petit rectangle noir pour terminer une preuve de theoreme, etc.
\def\eqdef {\buildrel \rm def \over =}    % Egal par definition.
\def\eqas  {\buildrel \rm a.s. \over =}   % Egal a.s.
\def\eqps  {\buildrel \rm p.s. \over =}   % Egal p.s.
\def\toas  {\buildrel \rm a.s. \over \to} % ---> a.s.
\def\tops  {\buildrel \rm p.s. \over \to} % ---> p.s.
\def\_{{\tt\char'137}}   %  Signe _ ("souligne").
\def\bs{{\tt\char'134}}  %  Signe \
\def\tttilde{{\tt\char'176}}  %  Signe ~
\def\<{$\langle$}        %  Signe <
\def\>{$\rangle$}        %  Signe >
\def\q{$\kern1.4em$}     % Space for indentation in slides and programs.

\def\version    {\begin{flushright} \it  Draft Version: \today \end{flushright}}
\def\workingpaper {\begin{flushright} \it WORKING PAPER. \end{flushright}}
\def\submitted#1 {\begin{flushright} \it SUBMITTED TO: #1.\end{flushright}}

\def\adrmaison {\begin {verse}
  Pierre L'Ecuyer\\
  45 Hampton Gardens\\ Pointe Claire (Qu\'e.) \\ H9S 5B8\\ CANADA \\
  T\'el.: (514) 426-4029 \end {verse}}
\def\adrum {\begin {verse}
  Professeur Pierre L'Ecuyer\\
  D\'epartement d'I.R.O., Universit\'e de Montr\'eal\\
  C.P.\ 6128, Succ.\ Centre-Ville, Montr\'eal, H3C 3J7\\ CANADA \\
  T\'el. : (514) 343-2143 \hspace{1in}
  FAX    : (514) 343-5834\\
  e-mail : {\tt lecuyer@IRO.UMontreal.ca}\\
  url    : {\tt www.iro.umontreal.ca/$\sim$lecuyer} \end {verse}}
%
\def\adrsalzburg {\begin {verse}
  Professor Pierre L'Ecuyer\\
  Institut f\"ur Mathematik\\ Universit\"at Salzburg\\
  Hellbrunnerstrasse 34, A-5020 Salzburg, AUSTRIA  \end {verse}}
\def\adrncsu {\begin {verse}
  Professor Pierre L'Ecuyer\\
  Department of Industrial Engineering\\
  North Carolina State University\\ Box 7906\\
  Raleigh, North Carolina 27695-7906, U.S.A. \\
  e-mail : {\tt lecuyer@IRO.UMontreal.ca}\\
  url    : {\tt www.iro.umontreal.ca/$\sim$lecuyer} \end {verse}}
%
\def\adrum0 {\begin {verse}
  Professeur Pierre L'Ecuyer\\
  D\'epartement d'I.R.O.\\ Universit\'e de Montr\'eal\\
  C.P.\ 6128, Succ.\ Centre-Ville\\ Montr\'eal, H3C 3J7\\ CANADA 
 \end {verse}}

% Redefinition du macro pour les ``footnotes''.
\catcode`\@=11
\def\ninepoint{\def\rm{\fam0\ninerm}
  \textfont0=\ninerm\normalbaselineskip=11pt
  \setbox\strutbox=\hbox{\vrule height8pt depth 3pt width0pt}%
  \normalbaselines\rm}
\def\vfootnote#1{\insert\footins\bgroup\ninepoint
  \interlinepenalty 100
  \leftskip=0pt \rightskip=0pt \spaceskip=0pt \xspaceskip=0pt
  \splittopskip=\ht\strutbox \floatingpenalty = 20000
  \splitmaxdepth=\dp\strutbox
  \item{#1}\footstrut\futurelet\next\fo@t}

% Macros pour inserer des bouts de code (programmes).
% Faire  \code ...  \endcode
{\obeyspaces\gdef {\ }}
\def\setverbatim{\def\par{\leavevmode\endgraf}
            \parskip=0pt\parindent=0pt\obeylines\obeyspaces }
\chardef\other=12
\def\ttverbatim{\setverbatim\tt
       \catcode`\{=\other \catcode`\}=\other \catcode`\_=\other
       \catcode`\^=\other \catcode`\$=\other \catcode`\%=\other
       \catcode`\#=\other \catcode`\&=\other \baselineskip=11pt
       }
    % Reproduit tel quel ce qui est ecrit, en caracteres \tt.
    % On doit faire  \begingroup\ttverbatim   ....  \endgroup
\def\smallttverbatim{\ttverbatim\small\tt}
\def\code {\vfil\vfilneg\vbox\bgroup\ttverbatim}
\def\longcode {\vfil\vfilneg\bgroup\ttverbatim}
\def\smallc {\small\tt\baselineskip=9.5pt}
\def\footc {\footnotesize\tt\baselineskip=9.0pt}
\def\smallcode {\code\smallc}
\let\endcode=\egroup
\let\vcode=\code
\let\endvcode=\egroup

%  Definition d'un module ou d'une classe.
\def\ps@nomark {\def\leftmark{} \def\rightmark{}}
\def\defmodule#1 {\addcontentsline{toc}{subsection}{#1} \markboth{#1}{#1}
   \centerline {\LARGE\bf #1}\bigskip \thispagestyle{nomark}}
\def\defclass#1 {\addcontentsline{toc}{subsection}{#1} \markboth{#1}{#1}
   \centerline {\LARGE\bf #1}\bigskip \thispagestyle{nomark}}
%  Lorsqu'on veut cacher certaines choses a l'usager, faire  \hide ... \endhide
\newif\iffull\fullfalse
\def\hide{\iffull}
\let\endhide=\fi

\def\parup{\nobreak\vskip -2pt\nobreak}

\def\tab{\small\dimen9=\parindent\parindent=0pt%
   \advance\leftskip by 1.5em\parup}
\def\tabb{\small\dimen9=\parindent\parindent=0pt%
   \advance\leftskip by 3.0em\parup}
\def\tabbb{\small\dimen9=\parindent\parindent=0pt%
   \advance\leftskip by 4.5em\parup}
\def\endtab{\vskip 0.01pt\advance\leftskip by -1.5em\normalsize%
   \parindent=\dimen9}
\def\endtabb{\vskip 0.01pt\advance\leftskip by -3.0em\normalsize%
   \parindent=\dimen9}
\def\endtabbb{\vskip 0.01pt\advance\leftskip by -4.5em\normalsize%
   \parindent=\dimen9}

% Pour mettre quelque chose dans une boite double.
\def\boxit#1{\vbox{\hrule height1pt
                   \hbox{\vrule width1pt\kern3pt
                         \vbox{\kern3pt#1\kern3pt
                              }\kern3pt\vrule width1pt
                        }\hrule height1pt }}
\def\boxr#1{\hfil\vbox{\hrule height1pt
                       \hbox{\vrule width1pt\kern3pt
                              \vbox{#1}\hfil
                              \kern3pt\vrule width1pt
                             }\hrule height1pt }}

% Synonymes plus courts pour  \begin{equation}, \begin{eqnarray}, etc.
\def\eq{\equation}  \def\endeq{\endequation}
\def\eqs{\eqnarray} \def\endeqs{\endeqnarray}
\def\eqsn{\begin {eqnarray*}} \def\endeqsn{\end{eqnarray*}}

% Proof avec boite alignee a droite a la fin.
\newenvironment{myproof}{{\em Proof.}}{\hspace*{\fill}$\Box$}
% \newenvironment{proof}{{\em Proof.}}{\hspace*{\fill}$\Box$}

% Macros (avec switch) pour faire apparaitre les noms des labels dans les eqs.
% Utiliser \eqlabel au lieu de \label.  Doit laisser un espace apres le }
% Pour que les etiquettes apparaissent, faire \seeeqlabelstrue  au debut.
\newif\ifseeeqlabels\seeeqlabelsfalse
\newbox\eqlab \setbox\eqlab=\hbox {}
\def\eqlabel#1 {\global\setbox\eqlab=\hbox
   {\ifseeeqlabels {\rm (#1)} \else {} \fi } \label{#1} }
\def\@eqnnum {{\rm \box\eqlab \setbox\eqlab=\hbox {} (\theequation)}}

% Comme ci-haut pour \eqlabel, mais pour les noms des autres labels
% (Theoremes, Propositions, etc.).
% Utiliser \vislabel au lieu de \label.
% Pour que les etiquettes apparaissent, faire \seevislabelstrue  au debut.
\newif\ifseevislabels\seevislabelsfalse
\def\vislabel#1 {\ifseevislabels {\ \em (#1).\ } \else {} \fi \label{#1} }

% Pour mettre des remarques temporaires.
\newif\ifREM\REMfalse
\def\REM#1 {\ifREM  \begin{quote} \small\em #1 \end{quote} \else {\null} \fi }

% Pour avoir "running head" et no. de page en haut de page, faire  \mytwoheads
% Si on veut la date en haut de chaque page, on fait aussi  \dateheadtrue
\newif\ifdatehead\dateheadfalse
\def\mytwoheads {\pagestyle{headings}
  \topmargin=-0.4in\headheight=0.2in\headsep=0.4in
  \oddsidemargin=0.2in\evensidemargin=0in
  \def\@evenhead {{\large\bf\thepage}\quad\leftmark\hfil
    \ifdatehead\small\it\today\fi}%         Left heading
  \def\@oddhead {\ifdatehead{\small\it\kern-1em\today}\fi\hfil
    \rightmark\quad\large\bf\thepage}}%     Right heading

\catcode`\@=12
