\defmodule {mystr}

This module offers some tools for the manipulation of
character strings. 

\bigskip\hrule
\code\iffalse
/* mystr.h for ANSI C */

#ifndef MYSTR_H
#define MYSTR_H
\fi

void mystr_Delete (char S[], unsigned int index, unsigned int len);
\endcode
 \tab  Deletes {\tt len} characters from S, starting at position
 {\tt index}.
 \endtab
\code


void mystr_Insert (char Res[], char Source[], unsigned int Pos);
\endcode
 \tab  Inserts the string {\tt Source} into {\tt Res}, 
  starting at position {\tt Pos}.
 \endtab
\code


void mystr_ItemS (char R[], char S[], const char T[], unsigned int N);
\endcode
 \tab  Returns in R the N-th substring of S (counting from 0).
  Substrings are delimited by any character from the set T.
 \endtab
\code


int mystr_Match (char Source[], char Pattern[]);
\endcode
 \tab  Returns {\tt 1} if the string {\tt Source} matches the 
  string {\tt Pattern}, and {\tt 0} otherwise.
  The characters ``?'' and ``*'' are recognized as wild characters in the
  string {\tt Pattern}.
 \endtab
\code


void mystr_Slice (char R[], char S[], unsigned int P, unsigned int L);
\endcode
 \tab  Returns in {\tt R} the substring in {\tt S} beginning at
  position {\tt P} and of length {\tt L}.
 \endtab
\code


void mystr_Subst (char Source[], char OldPattern[], char NewPattern[]);
\endcode
 \tab  Searches for the string {\tt OldPattern} in the string {\tt Source}, 
 and replaces its first occurence with {\tt NewPattern}.
 \endtab
\code


void mystr_Position (char Substring[], char Source[], unsigned int at,
                     unsigned int * pos, int * found);
\endcode
 \tab  Searches for the string {\tt Substring} in the string {\tt Source},
 starting at position {\tt at}, and returns the position of its first
 occurence in {\tt pos}.
 \endtab
\code
\iffalse

#endif
\fi\endcode
